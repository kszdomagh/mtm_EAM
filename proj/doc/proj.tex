\documentclass[oneside]{article}

\usepackage[T1]{fontenc}
\usepackage[utf8]{inputenc}
\usepackage{polski}
\usepackage{fancyhdr}
\usepackage{geometry}
\usepackage{multicol}
\usepackage{lastpage}
\usepackage{graphicx}
\usepackage{amsmath}

\geometry{margin=2cm}

\pagestyle{fancy}
\fancyhf{}
\fancyhead[L]{AGH WEAIiIB MTM}
\fancyhead[R]{\thepage\ /\ \pageref{LastPage}}
\fancyhead[C]{\today}

\begin{document}

% --- Tabela tytułowa ---
\begin{table}[h]
\centering
\begin{tabular}{lcr}
sem. V                  & \textbf{Elektroniczna Aparatura Medyczna} & Autorzy:  \\
Grupa I                 & Raport z projektu                      & Krzysztof Domański (419630)\\
(pon. 13:15)            &                                           & Igor Głowacz (419808) \\ 
                        &                                           & \\
\hline
\end{tabular}
\end{table}

% ===============           TYTUL SPRAWOZDANIA
\vspace{-15pt}
\begin{center}
    {\LARGE \textbf{Wzmacniacz biopotencjałów}}
\end{center}

% ===============           TRESC 




% ===============           TABLE OF CONTENTS
\tableofcontents

% ===============           ROZDZIAL 1 
\section{Wprowadzenie}
\noindent Celem projektu jest skonstruowanie wzmacniacza biopotencjałów, składającego się ze:
\begin{itemize}
    \item stopnia wejściowego opartego na specjalizowanym wzmacniaczu pomiarowym o konfigurowalnym wzmocnieniu,
    \item wstępnego filtru pasmowo-przepustowego,
    \item bardziej selektywnego filtru pasmowo-przepustowego,
    \item filtru typu "notch".
\end{itemize}

\noindent Nasza grupa otrzymała wariant drugi selektywnego filtru pasmowo-przepustowego, tj. filtr typu \textbf{Butterworth} dla pasma \textbf{1Hz – 300Hz}.

\begin{figure}[h]
    \centering
    \includegraphics[width=0.8\linewidth]{proj//png/ogolny.png}
    \caption{Schemat blokowy projektu wzmacniacza biopotencjałów.}
\end{figure}










% ===============           ROZDZIAL 1 
\clearpage
\twocolumn
\section{Konstrukcja pierwszego filtru pasmowo przepustowego}

Pierwszy filtr pasmowo-przepustowy ma posiadać wzmocnienie 10 oraz pasmo przepustowe w zakresie od 0.5 Hz do 1 kHz.

\begin{figure}[h]
    \centering
    \includegraphics[width=0.6\linewidth]{proj/png/1BP_draw.png}
    \caption{Schemat pierwszego filtru typu BANDPASS}
\end{figure}
\vspace{-13pt}
\subsection{Wstępne obliczenia}
\noindent Pierwszy filtr pasmowo-przepustowy posiada dwa bieguny, jeden biegun związany jest ze stałą czasową związaną z $C_1$ oraz $R_1$ na wejściu nieodwracającym wzmacniacza operacyjnego. Drugi biegun związany jest z komponentami $R_3$ oraz $C_2$. Wartości idealne zostały podane w treści instrukcji do projektu.

\begin{table}[h]
    \centering
    \begin{tabular}{|c|c|c|c|c|c|}
    \hline
                     & $R_1$   & $R_2$ & $R_3$ & $C_1$ & $C_2$  \\
    \hline
    Idealna wartość: & 330k    & 1k    &  100k &  1uF  & 3.3nF  \\
    \hline
    Użyta wartość:   & 326k    & 984    &  99k &  0.9uF  & 3.1nF  \\
    \hline
    \end{tabular}
    \caption{Wartości komponentów dla pierwszego filtru}
\end{table}
\noindent Obliczenia dla częstotliwości granicznych:
\begin{align*}
    f_d &= \frac{1}{2\pi C_1R_1} = \frac{1}{2\pi * 1u  * 330k} = 0.4822 \text{ Hz} \approx 0.5 \text{ Hz} \\
    f_g &= \frac{1}{2\pi C_2 R_3} = \frac{1}{2\pi * 3.3n  * 100k} = 488.28 \text{ Hz} \approx 500 \text{ Hz}
    \\
\end{align*}

\noindent Poniżej przedstawiono symulowaną charakterystykę filtru z zaznaczonymi punktami -3db.
\begin{figure}[h]
    \centering
    \includegraphics[width=1\linewidth]{proj//png/BP1_sim.png}
    \vspace{-14pt}
    \caption{Symulowana charakterystyka filtru}
\end{figure}


\newpage
\subsection{Przebiegi uzyskane w czasie symulacji oraz pomiarów}





% ===============           ROZDZIAL 2 
\clearpage
\section{Konstrukcja drugiego filtru pasmowo przepustowego}














% ===============           ROZDZIAL 3 
\clearpage
\section{Konstrukcja filtru typu notch}



















% ===============           ROZDZIAL 4 
\clearpage
\section{Montaż i konfiguracja stopnia wejściowego}



















% ===============           ROZDZIAL 5 
\clearpage
\section{Charakteryzacja całego toru pomiarowego}















% ===============           ROZDZIAL 6 
\clearpage
\section{Prezentacja działania wzmacniacza biopotencjałów}



\end{document}