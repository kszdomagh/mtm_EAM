\documentclass[oneside]{article}

\usepackage[T1]{fontenc}
\usepackage[utf8]{inputenc}
\usepackage{polski}
\usepackage{fancyhdr}
\usepackage{geometry}
\usepackage{multicol}
\usepackage{lastpage}
\usepackage{graphicx}
\usepackage{amsmath}

\geometry{margin=2cm}

\pagestyle{fancy}
\fancyhf{}
\fancyhead[L]{AGH WEAIiIB MTM}
\fancyhead[R]{\thepage\ /\ \pageref{LastPage}}
\fancyhead[C]{\today}

\begin{document}

% --- Tabela tytułowa ---
\begin{table}[h]
\centering
\begin{tabular}{lcr}
sem. V                  & \textbf{Elektroniczna Aparatura Medyczna} & Autorzy:  \\
Grupa I                 & Raport z projektu                      & Krzysztof Domański (419630)\\
(pon. 13:15)            &                                           & Igor Głowacz (419808) \\ 
                        &                                           & \\
\hline
\end{tabular}
\end{table}

% ===============           TYTUL SPRAWOZDANIA
\vspace{-15pt}
\begin{center}
    {\LARGE \textbf{Wzmacniacz biopotencjałów}}
\end{center}

% ===============           TRESC 
\begin{multicols}{2}




% ===============           TABLE OF CONTENTS
\tableofcontents

% ===============           ROZDZIAL 1 
\newcolumn
\section{Wprowadzenie}


\noindent Celem projektu jest skonstruowanie wzmacniacza biopotencjałów, składającego się ze:
\begin{itemize}
    \item stopnia wejściowego opartego na specjalizowanym wzmacniaczu pomiarowym o konfigurowalnym wzmocnieniu,

    \item wstępnego filtru pasmowo-przepustowego,

    \item bardziej selektywnego filtru pasmowo-przepustowego,

    \item filtru typu "notch".
\end{itemize}

\noindent Nasza grupa otrzymała wariant drugi selektywnego filtru pasmowo-przepustowego, tj. filtr typu \textbf{Butterworth} dla pasma \textbf{1Hz – 300Hz}.

\begin{center}
    \includegraphics[width=0.95\linewidth]{proj//png/ogolny.png}
    Schemat blokowy projektu wzmacniacza biopotencjałów.
\end{center}









% ===============           ROZDZIAL 1 
\clearpage
\section{Konstrukcja pierwszego filtru pasmowo przepustowego}









% ===============           ROZDZIAL 2 
\clearpage
\section{Konstrukcja drugiego filtru pasmowo przepustowego}














% ===============           ROZDZIAL 3 
\clearpage
\section{Konstrukcja filtru typu notch}



















% ===============           ROZDZIAL 4 
\clearpage
\section{Montaż i konfiguracja stopnia wejściowego}



















% ===============           ROZDZIAL 5 
\clearpage
\section{Charakteryzacja całego toru pomiarowego}















% ===============           ROZDZIAL 6 
\clearpage
\section{Prezentacja działania wzmacniacza biopotencjałów}


\end{multicols}

\end{document}