\documentclass[oneside]{article}

\usepackage[T1]{fontenc}
\usepackage[utf8]{inputenc}
\usepackage{polski}
\usepackage{fancyhdr}
\usepackage{geometry}
\usepackage{multicol}
\usepackage{lastpage}

\geometry{margin=2.5cm}

\pagestyle{fancy}
\fancyhf{}
\fancyhead[L]{AGH WEAIiIB MTM}
\fancyhead[R]{\thepage\ /\ \pageref{LastPage}}
\fancyhead[C]{\today}

\begin{document}

% --- Tabela tytułowa ---
\begin{table}[h]
\centering
\begin{tabular}{lcr}
sem. V                  & \textbf{Elektroniczna Aparatura Medyczna} & Autorzy:  \\
Grupa I                 & Raport z Laboratorium                      & Krzysztof Domański (419630)\\
(pon. 13:15)            &                                           & Igor Głowacz (XXXXXX) \\ 
                        &                                           & \\
\hline
\end{tabular}
\end{table}

% ===============           TYTUL SPRAWOZDANIA
\begin{center}
    {\LARGE \textbf{Wtórnik napięciowy, wzmacniacz nieodwracający, wzmacniacz odwracający}}
\end{center}

% ===============           TRESC 
\begin{multicols}{2}

% ===============           ROZDZIAL 1 
\section{Wtórnik napięcia}
\subsection{Teoretyczne wzmocnienie układu}
\subsection{Charakterystyka przejściowa}
\subsection{Slew Rate}
\subsection{Wpływ poziomu napięcia zasilania na parametry działania układu}

% ===============           ROZDZIAL 2 
\newpage
\section{Wzmacniacz nieodwracający}
\subsection{Teoretyczne wzmocnienie układu}
\subsection{Charakterystyka przejściowa}
\subsection{Charakterystyka częstotliwościowa}

% ===============           ROZDZIAL 3 
\newpage
\section{Filtr dolnoprzepustowy (układ całkujący) oparty o wzmacniacz odwracający}
\subsection{Teoretyczne wzmocnienie układu}
\subsection{Charakterystyka przejściowa}
\subsection{Charakterystyka częstotliwościowa}
\subsection{Przykładowe przebiegi}

% ===============           ROZDZIAL 3 
\section{Wnioski}

\end{multicols}

\end{document}